% T�m� on fysiikan laboratoriot�iden selostuspohjapohja.
% Pohja ei kuitenkaan ole mik��n virallinen ja oikea totuus, 
% eli muistakin pohjia voi k�ytt��. T�m�n tarkoitus on ainoastaan 
% auttaa opiskelijoita LaTeXin alkuun.
%
% Pohjaa saa levitt�� ja muuttaa vapaasti. Pohjan muuttajan toivotaan 
% kuitenkin lis��v�n tiedot muutoksesta ja sen ajankohdasta t�m�n 
% kommenttiosuuden loppuun.  
%
% Pikainen k�ytt�ohje:
% P�ivita kansilehden tiedot
% Kirjoita selostuksesi
% ATK-keskuksessa k��nn�t koodin seuraavilla komennoilla:
% use latex
% latex selkkari.tex
% 
% Esikatsele:
% xdvi selkkari
%
% Possuksi (tiedostonimeksi selostus.ps):
% dvips -o selostus.ps selkkari
%
% Toivottavasti t�m� pohja auttaa alkuun
%
% Jukka Katainen, 6.2.2004
%
\documentclass[a4paper,11pt]{article}

\frenchspacing
\usepackage[finnish]{babel}
\usepackage[latin1]{inputenc}
\usepackage[dvips]{graphicx}
\usepackage[T1]{fontenc}

\begin{document}

%T�st� alkaa selkkarin kansilehti. Vaihda tilalle tarvittavat tiedot.
\begin{titlepage}
\pagestyle{empty}
\begin{center}

\vspace*{3cm}
\noindent\LARGE{\textbf{Ty� 
%
%*******************************************************
%Ty�n numero:
99
%*******************************************************
%
\\
%
%*******************************************************
%Ty�n nimi:
Ty�n nimi
%*******************************************************
%
}}\\
\vspace*{2cm}
Ty�vuoro \LARGE{\textbf{
%
%*******************************************************
%Ty�vuoro:
99
%*******************************************************
%
}} pari \LARGE{\textbf{
%
%*******************************************************
%Parin numero:
9
%*******************************************************
}}\\
\vspace*{1cm}
\large{
\begin{tabular}{l l}
%
%*******************************************************
%Parin opiskelijat ja opiskelijanumerot:
Teemu Teekkari & 99999A\\
Tiina Teekkari & 88888B\\
%*******************************************************
%
\end{tabular}

\vspace*{1cm}
Selostuksen laati \emph{
%
%*******************************************************
% Selostuksen tekij�:
Teemu Teekkari
%*******************************************************
%
} \\
\vspace*{1cm}
\begin{tabular}{l l}
Mittaukset suoritettu & \textbf{
%
%*******************************************************
% Mittaukset suoritettu
pp.kk.vvvv
%*******************************************************
%
}\\
Selostus palautettu & \textbf{
%
%*******************************************************
% Selostus palautettu
pp.kk.vvvv
%*******************************************************
%
}\\
\end{tabular}
}
\end{center}
\end{titlepage}
%kansilehti loppuu t�h�n

%Varsinainan selkkari alkaa
\section{Johdanto}

\section{Laitteisto ja menetelm�t}

\section{Tulokset}

\section{Yhteenveto ja pohdinnat}


%Kirjallisuuviitteet
\begin{thebibliography}{99}
% Kirjallisuus viitteet t�h�n tapaan:
\item{R.W. Robinnet, Quantum Mechanics, Oxford University Press, 1997}
\label{robinnet}

\end{thebibliography}

%liitteet numeroituna
\section*{Liitteet}
\begin{enumerate}
%Liitteet t�h�n tapaan
\item{Mittausp�yt�kirja}\label{mittaus}

\end{enumerate}

\end{document}
