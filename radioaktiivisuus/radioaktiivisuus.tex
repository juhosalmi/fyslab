% Tämä on fysiikan laboratoriotöiden selostuspohjapohja.
% Pohja ei kuitenkaan ole mikään virallinen ja oikea totuus, 
% eli muistakin pohjia voi käyttää. Tämän tarkoitus on ainoastaan 
% auttaa opiskelijoita LaTeXin alkuun.
%
% Pohjaa saa levittää ja muuttaa vapaasti. Pohjan muuttajan toivotaan 
% kuitenkin lisäävän tiedot muutoksesta ja sen ajankohdasta tämän 
% kommenttiosuuden loppuun.  
%
% Pikainen käyttöohje:
% Päivita kansilehden tiedot
% Kirjoita selostuksesi
% ATK-keskuksessa käännät koodin seuraavilla komennoilla:
% use latex
% latex selkkari.tex
% 
% Esikatsele:
% xdvi selkkari
%
% Possuksi (tiedostonimeksi selostus.ps):
% dvips -o selostus.ps selkkari
%
% Toivottavasti tämä pohja auttaa alkuun
%
% Jukka Katainen, 6.2.2004
%
\documentclass[a4paper,11pt]{article}

\frenchspacing
\usepackage[finnish]{babel}
\usepackage[utf8]{inputenc}
\usepackage[dvips]{graphicx}
\usepackage[T1]{fontenc}

\begin{document}

%Tästä alkaa selkkarin kansilehti. Vaihda tilalle tarvittavat tiedot.
\begin{titlepage}
\pagestyle{empty}
\begin{center}

\vspace*{3cm}
\noindent\LARGE{\textbf{Työ 
%
%*******************************************************
%Työn numero:
55
%*******************************************************
%
\\
%
%*******************************************************
%Työn nimi:
Radioaktiivisuus ja säteily
%*******************************************************
%
}}\\
\vspace*{2cm}
Työvuoro \LARGE{\textbf{
%
%*******************************************************
%Työvuoro:
51
%*******************************************************
%
}} pari \LARGE{\textbf{
%
%*******************************************************
%Parin numero:
4
%*******************************************************
}}\\
\vspace*{1cm}
\large{
\begin{tabular}{l l}
%
%*******************************************************
%Parin opiskelijat ja opiskelijanumerot:
Juho Salmii & 80391C\\
Jukka Kemppainen & \\
%*******************************************************
%
\end{tabular}

\vspace*{1cm}
Selostuksen laati \emph{
%
%*******************************************************
% Selostuksen tekijä:
Juho Salmi
%*******************************************************
%
} \\
\vspace*{1cm}
\begin{tabular}{l l}
Mittaukset suoritettu & \textbf{
%
%*******************************************************
% Mittaukset suoritettu
11.11.2013
%*******************************************************
%
}\\
Selostus palautettu & \textbf{
%
%*******************************************************
% Selostus palautettu
18.11.2013
%*******************************************************
%
}\\
\end{tabular}
}
\end{center}
\end{titlepage}
%kansilehti loppuu tähän

%Varsinainan selkkari alkaa
\section{Johdanto}

Atomit koostuvat sen ytimeen pakkautuneista protoneista ja neutroneista sekä ulkokehällä sijaitsevistä elektroneista. Ytimen hiukkasten välillä on vahva vuorovaikutus, joka pitää atomiydintä koossa. Sähkömagneettinen vuorovaikutus saa puolestaan positiivisesti varautuneet ytimen protonit hylkimään toisiaan. Näiden voimien yhteisvaikutuksesta vain tietyn protoni- ja neutronimäärän sisältävät atomiytimet ovat stabiileja. \cite{ohje, wiki:radioaktiivisuus}

Epästabiilit atomiytimet pyrkivät stabiileiksi spontaanisti hajoamalla. Tätä kutsutaan radioaktiivisesksi hajoamiseksi. Hajoamisessa vapautuneiden hiukkasten sinkoutumista ympäristöön kutsutaan radioaktiiviseksi säteilyksi. \cite{ohje, wiki:radioaktiivisuus}

Tässä työssä tutkitaan alfa-, beeta- ja gammahajoamisia sekä -säteilyä. Alfahajoamisessa ydin emittoi kahden protonin ja neutronin muodostaman alfahiukkasen eli heliumytimen. Beetahajoamisessa protoni muuttuu neutroniksi tai päin vastoin. $\beta^-$-hajoamisessa ytimen neutroni muuttuu protoniksi vapauttaen elektronin ja antineutriinon. $\beta+$-hajoamisessa protoni muuttuu neutroniksi vapauttaen positronin ja neutriinon. Vapautuvia elektroneita tai positroneita kutsutaan beetahiukkasiksi ja -säteilyksi. \cite{ohje, wiki:radioaktiivisuus}

Alfa- ja beetahajoamisissa atomiydin voi jäädä virittyneeseen tilaan. Viritystilan lauetessa ydin emittoi viritysenergiansa gammafotonina. Tätä kutsutaan gammahajoamiseksi ja -säteilyksi. \cite{ohje, wiki:radioaktiivisuus}

Alfa- ja beetasäteilyä kutsutaan hiukkassäteilyksi, sillä niissä vapautuu hiukkasia, joilla on massa. Gammasätely on puolestaan sähkömagneettista säteilyä. \cite{ohje, wiki:radioaktiivisuus}

Eri radioaktiivisen säteilyn tyypeillä on erilaiset ominaisuudet \cite{ohje, wiki:radioaktiivisuus}. Tässä työssä tutustutaan alfa-, beeta- ja gammasäteilyn kantamaan ilmassa sekä läpäisykykyyn erilaisissa väliaineissa. 

\section{Laitteisto ja menetelmät}

Tämän luvun aliluvussa \ref{alfabeeta} tutkitaan alfa- ja beetasäteilyn ja aliluvussa \ref{gamma} gammasäteilyn kantamaa ilmassa sekä läpäisyä eri väliaineissa. 

Määritellään ja johdetaan työssä käytettävät termit ja fysikaaliset kaavat säteilyn kantaman ja läpäisyn mittaamiseksi. 

Säteilynilmaisin eli detektori on laite, joka tuottaa jännitepulssin havaitessaan säteilyhiukkasen. Pulssitaajuus ($\dot{n}$) on pulssimäärä ($n$) aikayksikköä kohden. Detektori ei havaitse kaikkia siihen osuvia hiukkasia. Havaitsemistodennäköisyyttä kutsutaan detektorin efektiivisyydeksi ($\epsilon$). Radioaktiivisten hajoamisten lukumäärä noudattaa Poisson-jakaumaa, joten pulssimäärän virhearviona voidaan käyttää sen neliöjuurta: $\Delta n = \sqrt{n}$.

Lähteen aktiivisuus $A$ voidaan määrittää mittaamalla lähteen säteilyn pulssitaajuutta ($\dot{n}$). Oletetaan lähteen koko etäisyyteen nähden niin pieneksi, että lähdettä voidaan käsitellä pistemäisenä. Oletetaan lisäksi detektori niin pieneksi, että sitä voidaan kuvata tasona. Gammasäteilyn kohdalla vuorovaikutustodennäköisyys ilman kanssa on niin pieni, ettei säteilyä absorboidu merkittävästi ennen sen osumista detektoriin. Tällöin detektorin havaitsema pulssitaajuus on 

\[ \dot{n} = \epsilon \cdot A \cdot n_A \cdot \frac{\Omega}{4 \pi} ,\]

jossa $n_A$ on yhdessä hajoamisessa emittoituvien gammakvanttien määrä, jossa lähde näkee detektorin. Detektorin pinta-ala on $a_d$, joten kaava voidaan kirjoittaa muotoon

\[ \dot{n} = \epsilon \cdot A \cdot n_A \cdot \frac{a_d}{4 \pi r^2} ,\]

jossa $r$ on detektorin etäisyys lähteestä. Pulssitaajuus on siis kääntäen verrannollinen etäisyyden neliöön. 

Säteilyn absorptiotodennäköisyys väliaineessa kasvaa väliaineen järjestysluvun funktiona ja pienenee fotonin energian funktiona. Gammasäteilyn intensiteett $\varphi$ vaimenee säteilyn väliaineessa kulkeman matkan $x$ funktiona eksponentiaalisesti:

\[ \varphi(x) = \varphi_0 \cdot e^{-\mu \cdot x} ,\]

jossa $\varphi_0$ on intensiteetti ennen väliaineeseen osumista. Matkavaimennuskerroin $\mu$ on väliaineesta ja fotonin energiasta riippuva absorptiotodennäköisyys pituusyksikköä kohden. Detektorin mittaama pulssitaajuus on suoraan verrannollinen säteilyn intensiteettiin: 

\[ \dot{n} = \epsilon \cdot \varphi \cdot a_d \]

\subsection{Alfa- ja beetasäteily}
\label{alfabeeta}

Alfa- ja beetasäteilyn mittauksessa säteilylähteet ja detektori on asennettu muovikoteloon, jossa säteilylähteiden etäisyyttä detektorista voidaan säätää ruuvilla ja säteilylähde voidaan vaihtaa tappia vetämällä tai työntämällä. Detektoriin kytketystä mittalaitteesta voidaan lisäksi valita mitattavaksi haluttu säteilylaji. Alfasäteilyn lähde koostuu Am-241-isotoopista ja beetasäteilyn ($\beta^-$) lähde Sr-90-isotoopista. 





\subsection{Gammasäteily}
\label{gamma}

\section{Tulokset}

\section{Yhteenveto ja pohdinnat}


%Kirjallisuuviitteet
\begin{thebibliography}{99}
% Kirjallisuus viitteet tähän tapaan:
%\item{R.W. Robinnet, Quantum Mechanics, Oxford University Press, 1997}

\bibitem{ohje} Harjoitustyöohje: Työ 55 Radioaktiivisuus ja säteily, http://physics.aalto.fi/pub/kurssit/Tfy-3.15xx/materiaali/55.pdf [Viitattu 18.11.2013]
\bibitem{wiki:radioaktiivisuus} Wikipedia-artikkeli radioaktiivisuudesta, http://fi.wikipedia.org/wiki/Radioaktiivisuus [Viitattu 18.11.2013]

\end{thebibliography}

%liitteet numeroituna
\section*{Liitteet}
\begin{enumerate}
%Liitteet tähän tapaan
\item{Mittauspöytäkirja}\label{mittaus}

\end{enumerate}

\end{document}
